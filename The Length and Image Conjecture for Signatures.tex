%% LyX 2.2.2 created this file.  For more info, see http://www.lyx.org/.
%% Do not edit unless you really know what you are doing.
\documentclass[12pt,english]{article}
\usepackage[T1]{fontenc}
\usepackage[latin9]{inputenc}
\usepackage{geometry}
\geometry{verbose,lmargin=3.5cm,rmargin=3.5cm}
\usepackage{amsmath}
\usepackage{amsthm}
\usepackage{amssymb}

\makeatletter
%%%%%%%%%%%%%%%%%%%%%%%%%%%%%% Textclass specific LaTeX commands.
\numberwithin{equation}{section}
  \theoremstyle{definition}
  \newtheorem{defn}{\protect\definitionname}[section]
  \theoremstyle{plain}
  \newtheorem{prop}{\protect\propositionname}[section]
  \theoremstyle{plain}
  \newtheorem{conjecture}{\protect\conjecturename}[section]
  \theoremstyle{plain}
  \newtheorem{lem}{\protect\lemmaname}[section]
  \theoremstyle{remark}
  \newtheorem{rem}{\protect\remarkname}[section]
  \theoremstyle{plain}
  \newtheorem{thm}{\protect\theoremname}[section]

%%%%%%%%%%%%%%%%%%%%%%%%%%%%%% User specified LaTeX commands.
\date{}

\makeatother

\usepackage{babel}
  \providecommand{\conjecturename}{Conjecture}
  \providecommand{\definitionname}{Definition}
  \providecommand{\lemmaname}{Lemma}
  \providecommand{\propositionname}{Proposition}
  \providecommand{\remarkname}{Remark}
\providecommand{\theoremname}{Theorem}

\begin{document}

\title{The Length/Image Conjecture for Signatures}
\maketitle

\section{The length conjecture for pure rough paths}

\subsection{A general upper estimate for pure rough paths}

Let $V$ be a Banach space and let $V^{\otimes n}$ be the induced
projective tensor products. Let $G^{(n)}(V)$ (respectively, $\mathfrak{g}^{(n)}(V)$)
be the degree $n$ free nilpotent Lie group (respectively, free nilpotent
Lie algebra) over $V$. Define $\mathcal{L}_{n}(V)$ to be the space
of degree $n$ homogeneous Lie polynomials over $V,$ so that 
\[
\mathfrak{g}^{(n)}(V)=\bigoplus_{k=1}^{n}\mathcal{L}_{k}(V).
\]

\begin{defn}
A \textit{pure $n$-rough path} is the weakly geometric $n$-rough
path of the form 
\[
\mathbf{X}_{t}=\exp\left(tl\right),
\]
where $l\in\mathfrak{g}^{(n)}(V)$, and the $n$-th homogeneous component
of $l$ is nonzero. Here the exponential is taken in the degree $n$
tensor algebra $T^{(n)}(V).$
\end{defn}
%
\begin{prop}
Let $\mathbf{X}_{t}=\exp(tl)$ ($0\leqslant t\leqslant1$) be a pure
$n$-rough path. Then the signature of $\mathbf{X}_{t}$ is 
\[
S(\mathbf{X})=\exp(l),
\]
where the expoenential is taken in the full tensor algebra $T((V))$.
\end{prop}
%
\begin{defn}
Let $\mathbf{X}_{t}$ be a $p$-rough path. We use $\mathbb{X}_{s,t}^{n}$
to denote the degree $n$ signature of $\mathbf{X}$ over $[s,t].$
Define the ``lim-sup'' functional 
\[
\widetilde{L}_{p}(\mathbf{X})\triangleq\overline{\lim_{n\rightarrow\infty}}\left((n/p)!\|\mathbf{\mathbb{X}}_{0,1}^{n}\|\right)^{p/n}.
\]
\end{defn}
%
\begin{conjecture}
\label{conj: length conjecture for pure rough path}Let $\mathbf{X}_{t}=\exp(tl)$
be a pure $m$-rough path, so that $l\in\mathfrak{g}^{(m)}(V).$ Let
$l_{m}\neq0$ be the $m$-th homogeneous component of $l$. Then 
\[
\|l_{m}\|_{V^{\otimes m}}=\widetilde{L}_{m}(\mathbf{X}).
\]
\end{conjecture}
%
We already know that the upper bound is true. Namely, we have the
following result.
\begin{prop}
\label{prop: upper bound}Using the same notation as before, we have
\[
\widetilde{L}_{m}(\mathbf{X})\leqslant\|l_{m}\|_{V^{\otimes m}}.
\]
\end{prop}
%
Using another notation, this upper bound can be written as: 
\begin{prop}
Let $\Vert\cdot\Vert_{n}$ be a cross-norm. Let $\mathcal{P}_{i}$
be order $i$ Lie polynomial (i.e. an element of $\mathcal{L}_{i}$).
Then 
\[
\limsup_{k}\Vert(\frac{k}{M})!\pi_{k}(\exp(\sum_{i=1}^{M}\mathcal{P}_{i}))\Vert_{k}^{\frac{M}{k}}\leq\Vert\mathcal{P}_{M}\Vert_{M}
\]
\end{prop}
To prove this, we first need a lemma. 
\begin{lem}
\label{lem:Binomial concentration}Let $0<\alpha<\beta\leq1$ and
let $a,b>0$. Then 
\[
\limsup_{k}\big((k\alpha)!\sum_{j=0}^{k}\frac{a^{j\alpha}b^{(k-j)\alpha}}{(j\beta)!((k-j)\alpha)!}\big)^{\frac{1}{k\alpha}}\leq b.
\]
\end{lem}
\begin{proof}
Let $\varepsilon>0$. It follows from Stirling's approximation that
for $0<\gamma\leq1$, 
\[
(j\gamma)!\sim(\frac{j\gamma}{e})^{j\gamma}\sqrt{2\pi j\gamma},
\]
and that $\beta>\alpha$, that there exists $J$ such that for all
$j\geq J$, 
\[
\frac{(j\alpha)!}{(j\beta)!}<\varepsilon^{j}.
\]
Therefore, 
\begin{eqnarray*}
 &  & \sum_{j=0}^{k}\frac{a^{j\alpha}b^{(k-j)\alpha}}{(j\beta)!((k-j)\alpha)!}\\
 & \leq & \sum_{j=0}^{J-1}\frac{a^{j\alpha}b^{(k-j)\alpha}}{(j\beta)!((k-j)\alpha)!}+\sum_{j=J}^{k}\frac{(\varepsilon^{1/\alpha}a)^{j\alpha}b^{(k-j)\alpha}}{(j\alpha)!((k-j)\alpha)!}.
\end{eqnarray*}
We may apply the neoclassical inequality to conclude that 
\begin{eqnarray}
 &  & \sum_{j=0}^{k}\frac{a^{j\alpha}b^{(k-j)\alpha}}{(j\beta)!((k-j)\alpha)!}\nonumber \\
 & \leq & \sum_{j=0}^{J-1}\frac{a^{j\alpha}b^{(k-j)\alpha}}{(j\beta)!((k-j)\alpha)!}+\frac{(\varepsilon^{1/\alpha}a+b)^{k\alpha}}{\alpha(k\alpha)!}.\label{eq:post neoclassical}
\end{eqnarray}
We observe that 
\[
(k\alpha)!\sum_{j=0}^{J-1}\frac{a^{j\alpha}b^{(k-j)\alpha}}{(j\beta)!((k-j)\alpha)!}\leq\begin{cases}
\frac{(k\alpha)!J(a\vee1)^{J\alpha}b^{(k-J)\alpha}}{((k-j)\alpha)!}, & \mbox{if }b\leq1;\\
\frac{(k\alpha)!J(a\vee1)^{J\alpha}b^{k\alpha}}{((k-j)\alpha)!}, & \mbox{if }b>1.
\end{cases}
\]
Noting that once again by Stirling's approximation, 
\[
\frac{(k\alpha)!}{((k-j)\alpha)!}\leq Ck^{J\alpha}
\]
where $C$ is a constant depending only on $j$ and $\alpha$. Then
\[
(k\alpha)!\sum_{j=0}^{J-1}\frac{a^{j\alpha}b^{(k-j)\alpha}}{(j\beta)!((k-j)\alpha)!}\leq\tilde{C}k^{J\alpha}b^{k\alpha}
\]
where $\tilde{C}$ depends on $J,a$ and $\alpha$. Therefore, 
\[
(k\alpha)!\sum_{j=0}^{k}\frac{a^{j\alpha}b^{(k-j)\alpha}}{(j\beta)!((k-j)\alpha)!}\leq\tilde{C}k^{J\alpha}b^{k\alpha}+\frac{(\varepsilon^{1/\alpha}a+b)^{k\alpha}}{\alpha}.
\]
We then take limsup in $k$ to obtain,
\begin{eqnarray*}
 &  & \limsup_{k}\big[(k\alpha)!\sum_{j=0}^{k}\frac{a^{j\alpha}b^{(k-j)\alpha}}{(j\beta)!((k-j)\alpha)!}\big]^{\frac{1}{k\alpha}}\\
 & \leq & \limsup_{k}\big[\tilde{C}k^{J\alpha}b^{k\alpha}+\frac{(\varepsilon^{1/\alpha}a+b)^{k\alpha}}{\alpha}\big]^{\frac{1}{k\alpha}}\\
 & = & \varepsilon^{1/\alpha}a+b.
\end{eqnarray*}
We then let $\varepsilon\rightarrow0$ to obtain 
\[
\limsup_{k}\big[(k\alpha)!\sum_{j=0}^{k}\frac{a^{j\alpha}b^{(k-j)\alpha}}{(j\beta)!((k-j)\alpha)!}\big]^{\frac{1}{k\alpha}}\leq b.
\]
\end{proof}
We now prove our main proposition. 
\begin{proof}
Note that 
\begin{eqnarray*}
 &  & \Vert\pi_{k}(\exp(\sum_{i=1}^{M}\mathcal{P}_{i}))\Vert\\
 & \leq & \sum_{N=0}^{\infty}\Vert\pi_{k}\big(\frac{(\sum_{i=1}^{M}\mathcal{P}_{i})^{\otimes N}}{N!}\big)\Vert\\
 & \leq & \sum_{N=0}^{\infty}\sum_{i_{1}+\ldots+i_{N}=k}\frac{\Vert\mathcal{P}_{i_{1}}\Vert\ldots\Vert\mathcal{P}_{i_{N}}\Vert}{N!}\\
 & = & \sum_{\sum in_{i}=k}\frac{\Vert\mathcal{P}_{1}\Vert^{n_{1}}\ldots\Vert\mathcal{P}_{M}\Vert^{n_{M}}}{(n_{1})!\ldots(n_{M})!}.
\end{eqnarray*}
If we substitute $\tilde{n}_{i}=in_{i}$, we have 
\begin{eqnarray}
 &  & \sum_{\sum in_{i}=k}\frac{\Vert\mathcal{P}_{1}\Vert^{n_{1}}\ldots\Vert\mathcal{P}_{M}\Vert^{n_{M}}}{(n_{1})!\ldots(n_{M})!}\nonumber \\
 & \leq & \sum_{\sum\tilde{n}_{i}=k}\frac{\Vert\mathcal{P}_{1}\Vert^{\tilde{n}_{1}}\ldots\Vert\mathcal{P}_{M}\Vert^{\tilde{n}_{M}/M}}{(\tilde{n}_{1})!\ldots(\tilde{n}_{M}/M)!}.\label{eq:First part}
\end{eqnarray}
Note first that we may use neoclassical inequality (the multi-factor
version, see Lemma 1 in \cite{FR11}) to bound 
\begin{eqnarray*}
 &  & \sum_{\sum_{i=1}^{M-1}\tilde{n}_{i}=k-\tilde{n}_{M}}\frac{\Vert\mathcal{P}_{1}\Vert^{\tilde{n}_{1}}\ldots\Vert\mathcal{P}_{M-1}\Vert^{\tilde{n}_{M-1}/(M-1)}}{(\tilde{n}_{1})!\ldots(\tilde{n}_{M-1}/(M-1))!}\\
 & \leq & \sum_{\sum_{i=1}^{M-1}\tilde{n}_{i}=k-\tilde{n}_{M}}\frac{\Vert\mathcal{P}_{1}\Vert^{\tilde{n}_{1}}\ldots\Vert\mathcal{P}_{M-1}\Vert^{\tilde{n}_{M-1}/(M-1)}}{(\tilde{n}_{1}/(M-1))!\ldots(\tilde{n}_{M-1}/(M-1))!}\\
 & \leq & \frac{(M-1)^{M-1}(\sum_{i=0}^{M-1}\Vert\mathcal{P}_{i}\Vert^{(M-1)/i})^{(k-\tilde{n}_{M})/(M-1)}}{((k-\tilde{n}_{M})/(M-1))!}.
\end{eqnarray*}
We now let 
\[
a=(\sum_{i=0}^{M-1}\Vert\mathcal{P}_{i}\Vert^{(M-1)/i}\big)^{M/(M-1)}.
\]
Then by continuing the calculation from (\ref{eq:First part}), 
\[
\Vert\pi_{k}(\exp(\sum_{i=1}^{M}\mathcal{P}_{i}))\Vert\leq(M-1)^{M-1}\sum_{\tilde{n}_{M}=0}^{k}\frac{a^{(k-\tilde{n}_{M})/M}\Vert\mathcal{P}_{M}\Vert^{\tilde{n}_{M}/M}}{(\frac{k-\tilde{n}_{M}}{M-1})!(\frac{\tilde{n}_{M}}{M})!}.
\]
We now apply Lemma \ref{lem:Binomial concentration} to see that 
\[
\limsup_{k}\Vert(\frac{k}{M})!\pi_{k}(\exp(\sum_{i=1}^{M}\mathcal{P}_{i}))\Vert^{\frac{M}{k}}\leq\Vert\mathcal{P}_{M}\Vert.
\]
\end{proof}
Therefore, the conjecture boils down to establishing the matching
lower bound.

\subsection{Algebraic development}

Again let $V$ be a Banach space and let $V^{\otimes n}$ be the induced
projective tensor products. Let $\mathfrak{h}$ be a fixed matrix
algebra over $\mathbb{R}$ or $\mathbb{C}$ (i.e. a sub-algebra of
$\mathfrak{gl}(k;\mathbb{R})$ or $\mathfrak{gl}(k;\mathbb{C})$ for
some $k\geqslant1$). Let $H$ be the corresponding Lie group embedded
as a subgroup of ${\rm GL}(k;\mathbb{R})$ or ${\rm GL}(k;\mathbb{C})$.
We equip $\mathfrak{h}$ with a norm induced by some given matrix
norm, whose precise choice is not so relevant at this point. Assume
that $\Phi:V\rightarrow\mathfrak{h}$ is a bounded linear functional.
Then for each $n\geqslant1,$ $\Phi$ induces a bounded linear functional
$\Phi^{(n)}:V^{\otimes n}\rightarrow\mathfrak{h}$, such that 
\[
\Phi(v_{1}\otimes\cdots\otimes v_{n})=\Phi(v_{1})\cdots\Phi(v_{n}).
\]
Since $V^{\otimes n}$ is the projective tensor product, we know from
\cite{LQ02}, Theorem 5.6.3 that
\[
\|\Phi^{(n)}\|_{{\rm op}}\leqslant\|\Phi\|_{{\rm op}}^{n}.
\]
In addition, $\Phi$ induces a homomorphism from the full tensor algebra
$T((V))$ to $\mathfrak{h},$ which is defined by (still denoted as
$\Phi$)
\[
\Phi\left(\sum_{n=0}^{\infty}\xi_{n}\right)\triangleq\sum_{n=0}^{\infty}\Phi^{(n)}(\xi_{n}),
\]
provided that the sum on the right hand side converges under the norm
on $\mathfrak{h}.$
\begin{defn}
Let $\mathbf{X}_{t}$ be a geometric rough path over $V$. The solution
to the differential equation 
\[
\begin{cases}
d\Gamma_{t}=\Gamma_{t}\Phi(d\mathbf{X}_{t}),\\
\Gamma_{0}=\mathrm{Id},
\end{cases}
\]
is called the \textit{Cartan development} of $\mathbf{X}_{t}$ onto
the Lie group $H$.
\end{defn}
%
By Picard iteration, it is easily seen that 
\begin{align*}
\Gamma_{t} & =\sum_{n=0}^{\infty}\int_{0<t_{1}<\cdots<t_{n}<t}\Phi(d\mathbf{X}_{t_{1}})\cdots\Phi(d\mathbf{X}_{t_{n}})\\
 & =\sum_{n=0}^{\infty}\Phi^{(n)}\left(\int_{0<t_{1}<\cdots<t_{n}<t}d\mathbf{X}_{t_{1}}\otimes\cdots\otimes d\mathbf{X}_{t_{n}}\right)\\
 & =\Phi\left(S(\mathbf{X})_{0,t}\right),
\end{align*}
where $S(\mathbf{X})_{0,t}$ is the signature of $\mathbf{X}$ over
$[0,t].$ Note that by the factorial decay of signature, $\Phi(S(\mathbf{X})_{0,t})\in\mathfrak{h}$
is well defined. In particular, if $g\triangleq S(\mathbf{X})_{0,1}$
is the signature of $\mathbf{X},$ then 
\[
\Gamma_{1}=\Phi(g).
\]

Now let $\mathbf{X}_{t}$ ($0\leqslant t\leqslant1$)be a weakly geometric
$p$-rough path. For each $\lambda>0,$ define $\delta_{\lambda}$
to be the dilation operator on the tensor algebra. Let $\Gamma_{t}^{\lambda}$
be the Cartan development of the dilated path $\delta_{\lambda}(\mathbf{X}_{t})$
onto $H$. Then 
\[
\Gamma_{t}^{\lambda}=\sum_{n=0}^{\infty}\lambda^{n}\int_{0<t_{1}<\cdots<t_{n}<t}\Phi(d\mathbf{X}_{t_{1}})\cdots\Phi(d\mathbf{X}_{t_{n}}).
\]

\begin{prop}
\label{prop: intermediate lower bound}Suppose that $\|\Phi\|_{{\rm op}}\leqslant1$.
Then we have 
\[
\overline{\lim_{\lambda\rightarrow\infty}}\frac{\log\|\Gamma_{1}^{\lambda}\|}{\lambda^{p}}\leqslant\widetilde{L}_{p}(\mathbf{X}).
\]
\end{prop}
\begin{proof}
The result can be proved in the same way as in \cite{BG18}, Proposition
4.1, together with the observation that 
\[
\lambda\mapsto e^{-\lambda^{p}\widetilde{L}_{p}(\mathbf{X})}\cdot\sum_{n=0}^{\infty}\frac{\lambda^{n}\widetilde{L}_{p}(\mathbf{X})^{n/p}}{(n/p)!}
\]
has polynomial growth in $\lambda$. 
\end{proof}

\subsection{Degree two pure rough paths in two dimension}

Consider $V=\mathbb{R}^{2},$ equipped with the $l^{1}$-norm with
respect to the standard basis $\{{\rm e}_{1},{\rm e}_{2}\}$. Also
equip the tensor products with the corresponding projective tensor
norms. 
\begin{lem}
$\|[{\rm e}_{1},{\rm e}_{2}]\|=2$.
\end{lem}
\begin{proof}
One simply observes that the projective tensor norms coincides with
the $l^{1}$-norm with respect to the standard tensor basis if $V$
is equipped with the $l^{1}$-norm.
\end{proof}
%
Consider the following pure $2$-rough path 

\[
\mathbf{X}_{t}=\exp\left(t(a{\rm e}_{1}+b{\rm e}_{2}+c[{\rm e}_{1},{\rm e}_{2}])\right),\ \ \ 0\leqslant t\leqslant1,
\]
where $a,b,c\in\mathbb{R}^{1}$ with $c\neq0,$ and the exponential
is taken in the degree two tensor algebra. Then the signature $g$
of $\mathbf{X}$ is given by 
\[
g=\exp\left(a{\rm e}_{1}+b{\rm e}_{2}+c[{\rm e}_{1},{\rm e}_{2}]\right)\in T((V)),
\]
where now the exponential is taken in the full tensor algebra. 

Now we choose the algebraic development in the following way. Let
$\mathfrak{h}=\mathfrak{gl}(2;\mathbb{C}),$ and equip $\mathfrak{h}\cong L(\mathbb{C}^{2};\mathbb{C}^{2})$
with the operator norm, where $\mathbb{C}^{2}$ is equipped with the
norm 
\[
\left|\left(\begin{array}{c}
z_{1}\\
z_{2}
\end{array}\right)\right|\triangleq|z_{1}|+|z_{2}|.
\]
 Define $\Phi:V\rightarrow\mathfrak{h}$ by 
\[
\Phi(x{\rm e}_{1}+y{\rm e}_{2})\triangleq\left(\begin{array}{cc}
x & iy\\
iy & -x
\end{array}\right).
\]

\begin{lem}
We have $\|\Phi\|_{{\rm op}}\leqslant1.$
\end{lem}
\begin{proof}
For $v=x{\rm e}_{1}+y{\rm e}_{2}$ and $z=\left(\begin{array}{c}
z_{1}\\
z_{2}
\end{array}\right)\in\mathbb{C}^{2},$ we have 
\begin{align*}
\Phi(x{\rm e}_{1}+y{\rm e}_{2})(z) & =\left(\begin{array}{cc}
x & iy\\
iy & -x
\end{array}\right)\cdot\left(\begin{array}{c}
z_{1}\\
z_{2}
\end{array}\right)=\left(\begin{array}{c}
xz_{1}+iyz_{2}\\
iyz_{1}-xz_{2}
\end{array}\right).
\end{align*}
Therefore, 
\begin{align*}
 & \left|\Phi(x{\rm e}_{1}+y{\rm e}_{2})(z)\right|\\
 & \leqslant|xz_{1}+iyz_{2}|+|iyz_{1}-xz_{2}|\\
 & \leqslant|x|\cdot|z_{1}|+|y|\cdot|z_{2}|+|y|\cdot|z_{1}|+|x|\cdot|z_{2}|\\
 & =(|x|+|y|)(|z_{1}|+|z_{2}|)\\
 & =|v|\cdot|z|.
\end{align*}
This implies that $\|\Phi\|_{{\rm op}}\leqslant1.$
\end{proof}
%
\begin{rem}
I don't think $\|\Phi\|_{{\rm op}}\leqslant1$ if $V=\mathbb{R}^{2}$
is equipped with the $l^{2}$-norm, no matter how we choose the norm
on $\mathbb{C}^{2}.$
\end{rem}
%

For each $\lambda>0$, let $\Gamma^{\lambda}$ be the Cartan development
of $\delta_{\lambda}(\mathbf{X}).$ Note that the signature of $\delta_{\lambda}(\mathbf{X})$
is 
\[
g^{\lambda}=\exp(\lambda a{\rm e}_{1}+\lambda b{\rm e}_{2}+\lambda^{2}c[{\rm e}_{1},{\rm e}_{2}]).
\]
It follows that 
\[
\Gamma_{1}^{\lambda}=\Phi(g^{\lambda})=\exp\left(\lambda aA+\lambda bB+\lambda^{2}c[A,B]\right),
\]
where 
\[
A\triangleq\Phi({\rm e}_{1})=\left(\begin{array}{cc}
1 & 0\\
0 & -1
\end{array}\right),\ B\triangleq\Phi({\rm e}_{2})=\left(\begin{array}{cc}
0 & i\\
i & 0
\end{array}\right).
\]
According to Proposition \ref{prop: upper bound} and Proposition
\ref{prop: intermediate lower bound}, we know that 
\[
\overline{\lim_{\lambda\rightarrow\infty}}\frac{\log\|\Gamma_{1}^{\lambda}\|}{\lambda^{2}}\leqslant\|c[{\rm e}_{1},{\rm e}_{2}]\|=2|c|.
\]

To estimate $\|\Gamma_{1}^{\lambda}\|,$ first recall a standard result
from functional analysis.
\begin{prop}
\label{prop: spectral bound}Let $T$ be a bounded linear operator
over a complex Banach space $W.$ Then 
\[
|\mu|\leqslant\|T\|_{{\rm op}}
\]
for every $\mu$ in the spectrum of $T.$
\end{prop}
%
Since $\|\Gamma_{1}^{\lambda}\|$ is the operator norm, Proposition
\ref{prop: spectral bound} tells us that estimating $\|\Gamma_{1}^{\lambda}\|$
boils down to eigenvalue estimates.

Now let 
\[
M(\lambda)\triangleq\lambda aA+\lambda bB+\lambda^{2}c[A,B].
\]
By straight forward calculation, 
\[
M(\lambda)=\left(\begin{array}{cc}
\lambda a & (\lambda b+2\lambda^{2}c)i\\
(\lambda b-2\lambda^{2}c)i & -\lambda a
\end{array}\right).
\]
It follows that the characteristic polynomial is 
\begin{align*}
{\rm ch}_{M(\lambda)}(t) & =\det(tI-M(\lambda))\\
 & =t^{2}-(\lambda a)^{2}+(\lambda b)^{2}-(2\lambda^{2}c)^{2}.
\end{align*}
Therefore, when $\lambda$ is large, $M(\lambda)$ has two distinct
real eigenvalues
\[
\pm\sqrt{4\lambda^{4}c^{2}+\lambda^{2}a^{2}-\lambda^{2}b^{2}}.
\]
In particular, $M(\lambda)$ is diagonalizable, and thus $\Gamma_{1}^{\lambda}=\exp(M(\lambda))$
has two distinct real eigenvalues
\[
\exp\left(\pm\sqrt{4\lambda^{4}c^{2}+\lambda^{2}a^{2}-\lambda^{2}b^{2}}\right).
\]
By Proposition \ref{prop: spectral bound}, we conclude that 
\[
\|\Gamma_{1}^{\lambda}\|\geqslant\exp\left(\sqrt{4\lambda^{4}c^{2}+\lambda^{2}a^{2}-\lambda^{2}b^{2}}\right).
\]
Therefore, 
\[
\overline{\lim_{\lambda\rightarrow\infty}}\frac{\log\|\Gamma_{1}^{\lambda}\|}{\lambda^{2}}\geqslant\overline{\lim_{\lambda\rightarrow\infty}}\frac{\sqrt{4\lambda^{4}c^{2}+\lambda^{2}a^{2}-\lambda^{2}b^{2}}}{\lambda^{2}}=2|c|.
\]
 In particular, we obtain the following result.
\begin{thm}
Under the $l^{1}$-norm on $\mathbb{R}^{2}$ with the induce projective
tensor norms, Conjecture \ref{conj: length conjecture for pure rough path}
holds for $2$ dimensional pure $2$-rough paths.
\end{thm}

Again for testing



\begin{thebibliography}{1}
\bibitem{BG18}H. Boedihardjo and X. Geng, tail asymptotics of the
Brownian signature, \textit{arXiv preprint}, 2018.

\bibitem{LQ02}T. Lyons and Z. Qian, \textit{System control and rough
path}s, Oxford University Press, 2002.
\end{thebibliography}

\end{document}
